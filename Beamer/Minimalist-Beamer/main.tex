% The template is suitable for Chinese and English.
% > To properly compile this template with bibliography support, please use the following sequence:
% >
% > ```
% > xelatex main.tex
% > bibtex main
% > xelatex main.tex
% > xelatex main.tex
% > ```



\documentclass[8pt]{beamer}

\input{init.tex}

\begin{document}

\sf

% There are two types of covers set up here, so you can choose one of them as needed

% \begin{frame}[plain]

%     \centering
%     \vspace{13mm}
%     \titlecolorbox{A Minimalist-style Slides}

%     \vspace{7mm}
%     \begin{tabular}{>{\color{themecolor}}r@{\hspace{3pt}}l}
%         Presenter & Zhang San \\
%         Supervisor & Li Si \\
%     \end{tabular}

%     \vspace{4mm} 
%     \today

% \end{frame}

\begin{frame}[plain]

    \setlength{\parskip}{0.2em}
    \vspace{13mm}
    {\color{themecolor}\huge 
    A Minimalist-style Slides
    }

    \begin{tikzpicture}
        \fill[black!50] (0,0) rectangle (0.98\textwidth, 0.5pt);
        \fill[themecolor] (0,0) rectangle (0.98\textwidth/2, 0.5pt);
    \end{tikzpicture}

\vspace{7mm}

    {\large
    San Zhang\textsuperscript{1}, 
    {\color{themecolor!80!black} Pinzhong Zheng}\textsuperscript{2}, 
    Si Li\textsuperscript{2}
    }
    
\vspace{4mm}
    
    \textsuperscript{1}The Chinese University of Hong Kong, Shenzhen
    
    \textsuperscript{2}The Hong Kong Polytechnic University, Hong Kong

\vspace{4mm}

    {\small
    PolyU, 2025
    }

    % \vspace{7mm}
    % \begin{tabular}{>{\color{themecolor}}l@{\hspace{3pt}}l}
    %     Presenter & Zhang San \\
    %     Supervisor & Li Si \\
    % \end{tabular}

    % \vspace{4mm} 
    % \today

\end{frame}


\begin{frame}
    \frametitle{Outline}
    \setlength{\parskip}{0.2em}
    \tableofcontents
\end{frame}

\section{Template Introduction}
  
\makesection

\subsection{Overview and File Structure}

\begin{frame}{\insertsection}{\insertsubsection}

    This presentation template is developed based on \lstinline|ctexbeamer| and features a fresh, minimalist design, making it well-suited for academic talks, thesis defenses, and related occasions. Its file structure is organized as follows:

    \begin{table}
        \renewcommand{\arraystretch}{1.5}
        \centering
        \begin{tabular}{ll}
            \toprule
            File Path & Description \\
            \midrule
            \lstinline|main.tex| & The main file that defines the overall structure and content \\
            \lstinline|init.tex| & The initialization script that loads components \\
            \lstinline|reference.bib| & A BibTeX file containing references \\
            \lstinline|init/cmds.tex| & Defines reusable custom commands \\
            \lstinline|init/code.tex| & Configures the styling for source code blocks \\
            \lstinline|init/color.tex| & Defines the color palette used in the slides \\
            \lstinline|init/format.tex| & Specifies formatting settings for elements \\
            \lstinline|init/pkg.tex| & Loads all necessary LaTeX packages \\
            \bottomrule
        \end{tabular}
    \end{table}

\end{frame}

\section{Preparation Work}

\makesection

\subsection{Template Setup Instructions}

\begin{frame}{\insertsection}{\insertsubsection}
    \begin{block}{Template Setup Instructions}
        If you are using the Southeast University theme, you can skip this section and directly apply the provided template. For other universities or institutions, please follow the instructions below to customize the template:
        \begin{enumerate}
            \item Save your institution’s logo at \lstinline|figs/logo.jpg|.
            \item In \lstinline|init/color.tex|, set \lstinline|themecolor| to your institution’s official theme color, and adjust \lstinline|themered| to a complementary accent color.
            \item Once these steps are completed, your template is ready to go!
        \end{enumerate}
    \end{block}
\end{frame}

\section{Using the Template}

\makesection

\subsection{Sections and Table of Contents}

\begin{frame}[fragile]{\insertsection}{\insertsubsection}

    \begin{block}{Creating Sections}
        \begin{enumerate}
            \item Use the \lstinline|\section{}| and \lstinline|\subsection{}| to define sections and subsections.
            \item After defining a section, \lstinline|\makesection[width]| can be used to generate a cover. The optional \lstinline|width| sets the length of the horizontal line; the default value is 0.4.
            \begin{lstlisting}[style=latex]
\section{Direct Usage}
\subsection{Sections and Table of Contents}
\makesection
            \end{lstlisting}
        \end{enumerate}
    \end{block}
    \vspace{-3mm}
    \begin{block}{Generating a Table of Contents}
        \begin{enumerate}
            \item The following code creates a slide containing the table of contents.
            \begin{lstlisting}[style=latex]
\begin{frame}
    \frametitle{Table of Contents}
    \setlength{\parskip}{0.2em} % Adjust line spacing between paragraphs
    \tableofcontents
\end(*@@*){frame}
            \end{lstlisting}
        \end{enumerate}
    \end{block}

\end{frame}


\subsection{Pages and Blocks}
\begin{frame}[fragile]{\insertsection}{\insertsubsection}
    \begin{block}{Using Pages}
        \begin{enumerate}
            \item The following command generates a slide where the main and subheadings are automatically set to the current section and subsection titles:
            \begin{lstlisting}[style=latex]
\begin{frame}{\insertsection}{\insertsubsection}
    % Slide content
\(*@@*)end{frame}
            \end{lstlisting}
            \item You can also define custom headings with \lstinline|\begin{frame}{title}{subtitle}|. The subtitle is optional. If both fields are left blank, no header will appear on the slide.
            \item For slides without a header or footer—commonly used for cover or closing slides—use \lstinline|\begin{frame}[plain]|.
            \item Use \lstinline|\begin{frame}[fragile]{title}{subtitle}| when the slide includes code blocks.
        \end{enumerate}
    \end{block}
\end{frame}

\begin{frame}[fragile]{\insertsection}{\insertsubsection}
    \begin{block}{Using Custom Blocks}
        \begin{enumerate}
            \item A block can be created using the following command:
            \begin{lstlisting}[style=latex]
\begin{block}{title}
    % block content
\end{block}
            \end{lstlisting}
            \item To customize the block's theme color, use the \lstinline|\colorlet{themecolor}{somecolor}| command before and after the block.
        \end{enumerate}
    \end{block}

    \colorlet{themecolor}{themered}
    \begin{block}{Block with Red Theme}
        \begin{enumerate}
            \item This is a block with its theme color set to red.
            \item The following snippet demonstrates how to define such a block.
            \begin{lstlisting}[style=latex]
\colorlet{themecolor}{themered}
\begin{block}{title}
    % block content
\end{block}
\colorlet{themecolor}{themegreen}
            \end{lstlisting}
        \end{enumerate}
    \end{block}
    \colorlet{themecolor}{themegreen}

\end{frame}

\subsection{Lists}

\begin{frame}{\insertsection}{\insertsubsection}

    \begin{block}{Using Lists}
        \begin{enumerate}
            \item Use the \lstinline|itemize| environment to create bulleted (unordered) lists.
            \item Use the \lstinline|enumerate| environment to create numbered (ordered) lists.
        \end{enumerate}
    \end{block}

    \begin{block}{List Examples}

        \begin{itemize}
            \item Example of a bulleted list.
            \item Another item in the bulleted list.
            \begin{itemize}
                \item A nested item in the bulleted list.
                \begin{itemize}
                    \item A second-level nested item in the bulleted list.
                    \item Another second-level nested item in the bulleted list.
                \end{itemize}
                \item Another nested item in the bulleted list.
            \end{itemize}
        \end{itemize}

        \begin{enumerate}
            \item Example of a numbered list.
            \item Another item in the numbered list.
            \begin{enumerate}
                \item A nested item in the numbered list.
                \begin{enumerate}
                    \item A second-level nested item in the numbered list.
                    \item Another second-level nested item in the numbered list.
                \end{enumerate}
                \item Another nested item in the numbered list.
            \end{enumerate}
        \end{enumerate}

    \end{block}

\end{frame}

\subsection{Emphasis and Annotations}

\begin{frame}[fragile]{\insertsection}{\insertsubsection}

    The table below presents various commands used for textual emphasis and annotation. Footnotes can be added using the command \lstinline|\footnote{}|\footnote{This is a footnote.}.
    \begin{table}
        \small
        \renewcommand{\arraystretch}{1.5}
        \setlength{\tabcolsep}{2mm}
        \begin{tabular}{llll}
            \toprule
            Command & Output & Command & Output\\
            \midrule
            \lstinline|\textsf{}| & \textsf{Sans-serif} & \lstinline|\textbf{}| & \textbf{Bold} \\
            \lstinline|\textrm{}| & \textrm{Serif} & \lstinline|\texttt{}| & \texttt{Monospaced} \\
            \lstinline|\uline{}| & \uline{Underline} & \lstinline|\uwave{}| & \uwave{Wavy underline} \\
            \lstinline|\sout{}| & \sout{Strikethrough} & \lstinline|\emph{}| & \emph{Italic emphasis} \\
            \lstinline|\highlight{}| & \highlight{Highlighted text} & \lstinline|\ulhighlight{}| & \ulhighlight{Underlined highlight} \\
            \lstinline|\stronghighlight{}| & \highlight{Strong highlight} & \lstinline|\ulhighlight{}| & \ulstronghighlight{Underlined strong highlight} \\
            \bottomrule
        \end{tabular}
    \end{table}
    \begin{myquote}
        This is a quotation environment, but I often use it as a note or aside.
    \end{myquote}

    \begin{callout}{Callout}
        This block can be used to clarify terms introduced in the slides or provide peripheral information that complements the main content. It has been filled with extra text to enhance its visual appeal.
    \end{callout}

\end{frame}


\subsection{Figures and Tables}

\begin{frame}[fragile]{\insertsection}{\insertsubsection}

    \begin{block}{Including Figures}
        \begin{enumerate}
            \item To include a figure, use the following command.
            \begin{lstlisting}[style=latex]
\begin{figure}
    \centering
    \includegraphics[width=0.6\textwidth]{example.jpg}
    \caption{An Example Figure}
    \label{fig:example}
\end{figure}
            \end{lstlisting}
            \begin{figure}
                \centering
                \includegraphics[width=0.5\textwidth]{figs/example.jpg}
                \caption{An Example Figure}
                \label{fig:example}
            \end{figure}
        \end{enumerate}
    \end{block}

\end{frame}


\begin{frame}[fragile]{\insertsection}{\insertsubsection}

    \begin{block}{Inserting Tables}
        \begin{enumerate}
            \item Use the following command to insert a table.
        \begin{lstlisting}[style=latex]
\begin{table}
    \centering
    \caption{Sample Table}\label{tab:example}
    \begin{tabular}{cc}
        \toprule
        % Header
        \midrule
        % Table Data
        \bottomrule
    \end{tabular}
\end{table}
        \end{lstlisting}

        \begin{table}
            \centering
            \caption{Sample Table}\label{tab:example}
            \begin{tabular}{cccc}
                \toprule
                Column 1 & Column 2 & Column 3 & Column 4 \\
                \midrule
                1 & 2 & 3 & 4 \\
                5 & 6 & 7 & 8 \\
                \bottomrule
            \end{tabular}
        \end{table}
        \end{enumerate}
    \end{block}
\end{frame}


\subsection{Equations and Code Blocks}

\begin{frame}[fragile]{\insertsection}{\insertsubsection}
    \label{frame:eq_and_code}

    \begin{block}{Using Equations}
        The following demonstrates how to insert an equation.
        \begin{equation}
            \label{eq:example}
            i\hbar\frac{\partial \psi}{\partial t}
= \frac{-\hbar^2}{2m} \left(
\frac{\partial^2}{\partial x^2}
+ \frac{\partial^2}{\partial y^2}
+ \frac{\partial^2}{\partial z^2}
\right) \psi + V \psi.
        \end{equation}
    \end{block}

    \begin{block}{Using Code Blocks}
        \begin{enumerate}
            \item Use the \lstinline|\lstlisting{}| command to insert inline code.
            \item The syntax and effect of inserting a code block are shown below.
            \begin{lstlisting}[style=latex]
\begin{lstlisting}[style=latex]
% Content of the code block
\end(*@@*){lstlisting}
            \end{lstlisting}
        \end{enumerate}
    \end{block}
\end{frame}


\subsection{References and Cross-Referencing}

\begin{frame}[fragile]{\insertsection}{\insertsubsection}

    \begin{block}{Managing References}
        \begin{enumerate}
            \item First, store your bibliographic data in the \lstinline|reference.bib| file.
            \item Use the \lstinline|\footfullcite{}| command to generate footnote-style citations.
            \item For example, Kopka et al. authored a well-known book on \LaTeX{}\footfullcite{kopka2003guide}.
        \end{enumerate}
    \end{block}

    \begin{block}{Using Cross-References}
        \begin{enumerate}
            \item To reference figures, tables, or equations, use the \lstinline|\ref{}| command.
            \item To cite the page number, use the \lstinline|\pageref{}| command.
            \item For instance, the equation on page \pageref{frame:eq_and_code} can be cited as equation (\ref{eq:example}).
        \end{enumerate}
    \end{block}

\end{frame}


\section{Conclusion}

\subsection{Conclusion}

\makesection

\begin{frame}{\insertsection}{\insertsubsection}
    \begin{block}{Conclusion}
        \begin{enumerate}
            \item We present a streamlined \LaTeX{} template.
            \item The template integrates essential features such as image, table, equation, and code block insertion.
            \item That concludes our presentation.
        \end{enumerate}
    \end{block}

    \begin{block}{Acknowledgements}
        \begin{enumerate}
            \item We sincerely thank everyone for their support.
            \item This template is based on the elegant slides\footnote{\url{https://www.overleaf.com/latex/templates/elegant-slides/yfqyhpprvdmg}} template.
        \end{enumerate}
    \end{block}
\end{frame}

\begin{frame}{References}
    \printbibliography
\end{frame}


\end{document}