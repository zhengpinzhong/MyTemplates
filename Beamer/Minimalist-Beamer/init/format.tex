% Beamer 模板设置
\setbeamertemplate{navigation symbols}{}
\setbeamertemplate{footline}[page number]
\setbeamertemplate{section in toc}[sections numbered]

% Itemize 格式设置
\setbeamertemplate{itemize item}[circle] % 项目符号设置为圆形
\setbeamertemplate{itemize subitem}{$\circ$} % 子项目符号设置为圆形
\setbeamertemplate{itemize subsubitem}[circle] % 子子项目符号设置为圆形

% Enumerate 格式设置
\setbeamertemplate{enumerate subitem}{\alph{enumii}.} % 子项目符号设置为小写字母
\setbeamertemplate{enumerate subsubitem}{\roman{enumiii}.} % 子子项目符号设置为罗马数字

% 颜色设置
\setbeamercolor{frametitle}{bg=, fg=themecolor}
\setbeamercolor{block title}{bg=themecolor!8, fg=themecolor}
\setbeamercolor{description item}{fg=themecolor}
\setbeamercolor{section in toc}{fg=themecolor, bg=}
\setbeamercolor{subsection in toc}{fg=black!60}
\setbeamercolor{caption name}{fg=themecolor}
\setbeamercolor{bibliography entry author}{fg=black}
\setbeamercolor{bibliography entry title}{fg=black}
\setbeamercolor{bibliography entry journal}{fg=black}
\setbeamercolor{bibliography entry note}{fg=black}
\setbeamercolor{bibliography item}{fg=themecolor}
\setbeamercolor{itemize item}{fg=themecolor} % 项目符号颜色设置
\setbeamercolor{itemize subitem}{fg=themecolor} % 子项目符号颜色设置
\setbeamercolor{itemize subsubitem}{fg=themecolor} % 子子项目符号颜色设置
\setbeamercolor{enumerate item}{fg=themecolor} % 枚举项颜色设置
\setbeamercolor{enumerate subitem}{fg=themecolor} % 枚举子项颜色设置
\setbeamercolor{enumerate subsubitem}{fg=themecolor} % 枚举子子项颜色设置

% 字体设置
\setsansfont{Arial}%\setsansfont{Calibri}
\setbeamerfont{frametitle}{size=\large}
\setbeamerfont{framesubtitle}{size=\normalsize}
\setbeamerfont{block title}{size=\large}
\setbeamerfont{section in toc}{size=\large}
\setbeamerfont{subsection in toc}{size=\normalsize}
\setbeamerfont{caption name}{size=\normalsize}
\setbeamerfont{caption}{size=\normalsize}
\setbeamerfont{footnote}{size=\tiny}
\setbeamerfont{description item}{size=\normalsize}

% 页脚模板设置
\setbeamertemplate{footline}{%
  \begin{beamercolorbox}[wd=\paperwidth, ht=2.25ex, dp=1ex, leftskip=0.5cm, rightskip=0.5cm]{title in head/foot}%
    \hfill{\textcolor{black!60}{\insertframenumber/\inserttotalframenumber}}\hspace*{1mm}\vspace*{2mm}
  \end{beamercolorbox}%
}

% 自定义标题模板
\defbeamertemplate*{frametitle}{}[1][]{
  \nointerlineskip%
  \vspace*{3mm}
  \hspace*{-1mm}
  \begin{beamercolorbox}[sep=0.3cm, wd=\paperwidth, leftskip=0.5cm, rightskip=0.5cm]{frametitle}
    {\usebeamerfont{frametitle}\insertframetitle}
    {\usebeamerfont{framesubtitle}\color{black!60}\insertframesubtitle
    \hfill
    \raisebox{-1mm}{\includegraphics[width=22mm]{figs/polyu.pdf}}
    }\par
    \vskip-1.5ex
    \begin{tikzpicture}[remember picture, overlay]
      \draw[line width=0.2pt] (0,0) -- (\textwidth+4mm,0);
    \end{tikzpicture}
  \end{beamercolorbox}
  \vspace*{-7mm}
} 
  
% 设置段落间距
\setlength{\parskip}{0.5em}

% 参考文献
\bibliography{refs.bib}
\normalem

% 数学字体主题设置
\usefonttheme[onlymath]{serif}

% 表格环境设置
\AtBeginEnvironment{table}{\setlength{\parskip}{0em}}
