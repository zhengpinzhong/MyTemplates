\documentclass[11pt]{article}

% --- 常用宏包 ---
\usepackage[utf8]{inputenc}
\usepackage[T1]{fontenc}
\usepackage{lmodern}
\usepackage{amsmath, amssymb, amsthm}
\usepackage{graphicx}
\usepackage{caption}
\usepackage{float}
\usepackage{booktabs}
\usepackage{cite}
\usepackage{geometry}
\usepackage{hyperref}
\usepackage{mathtools}
\usepackage{xcolor}

% 页面设置
\geometry{margin=1in}

% --- 自定义定理样式: italic remark ---
\newtheoremstyle{italicremark}  
  {3pt} {3pt} {\itshape} {} {\bfseries} {.} { } {}

% --- 定理环境定义 ---
\newtheorem{theorem}{Theorem}[section]
\newtheorem{lemma}[theorem]{Lemma}
\newtheorem{proposition}[theorem]{Proposition}
\newtheorem{corollary}[theorem]{Corollary}
\theoremstyle{definition}
\newtheorem{definition}[theorem]{Definition}
\theoremstyle{italicremark}
\newtheorem{remark}[theorem]{Remark}

% --- 自定义命令示例 ---
\newcommand{\e}{\mathrm{e}} % 定义自然对数底e
\newcommand{\ii}{\mathrm{i}} % 定义虚数单位i
\newcommand{\abs}[1]{\left\vert #1 \right\vert} % 定义绝对值符号
\newcommand{\norm}[1]{\left\| #1 \right\|} % 定义范数符号
\newcommand{\RR}{\mathbb{R}} % 定义实数集符号
\newcommand{\CC}{\mathbb{C}} % 定义复数集符号
\newcommand{\ZZ}{\mathbb{Z}} % 定义整数集符号
\newcommand{\NN}{\mathbb{N}} % 定义自然数集符号
\newcommand{\QQ}{\mathbb{Q}} % 定义有理数集符号
\newcommand{\EE}{\mathbb{E}} % 定义期望符号
\newcommand{\PP}{\mathbb{P}} % 定义概率符号
\newcommand{\FF}{\mathbb{F}} % 定义滤波器符号

\newcommand{\hI}{\mathcal{I}}
\newcommand{\hL}{\mathcal{L}}
\newcommand{\hN}{\mathcal{N}}
\newcommand{\Lk}{\mathcal{L}_{\kappa}}
\newcommand{\Linf}{L^{\infty}}
\newcommand{\rd}{\mathrm{~d}}
\newcommand{\nn}{\nonumber}

% --- 文章信息 ---
\title{A Minimal Template: Demonstration of LaTeX Environments}
\author{Pinzhong Zheng\thanks{The Hong Kong Polytechnic University, pinzhong.zheng@connect.polyu.hk}}
\date{\today}

\begin{document}

\maketitle

\begin{abstract}
This is a sample abstract. 
\end{abstract}

\section{Introduction}
Lorem ipsum dolor sit amet, consectetur adipiscing elit~\cite{DuEtAl2021MaximumBound}. 

\subsection{Sample Equation}
Consider the following equation:
\begin{equation}\label{eq:sample}
    \int_{\Omega} u(x) \, \mathrm{d}x = 0.
\end{equation}

\section{Mathematical Environments}

\begin{theorem}[Sample Theorem]\label{thm:sample}
Let $x \in \RR$. Then for all $x$, we have $\abs{x} \ge 0$.
\end{theorem}

\begin{proof}
Trivially, by the definition of absolute value.
\end{proof}

\begin{lemma}[Sample Lemma]\label{lem:sample}
For any $a, b \in \RR$, $\abs{a+b} \le \abs{a} + \abs{b}$.
\end{lemma}

\begin{proposition}[Sample Proposition]
If $x > 1$, then $x^2 > 1$.
\end{proposition}

\begin{corollary}[Sample Corollary]
If $x > 1$, then $x^4 > 1$.
\end{corollary}

\begin{definition}[Sample Definition]
A set $S$ is \emph{bounded} if there exists $M > 0$ such that $\abs{x} < M$ for all $x \in S$.
\end{definition}

\begin{remark}
This is a remark. Lorem ipsum dolor sit amet, consectetur adipiscing elit.
\end{remark}

\section{Figures and Tables}

\begin{figure}[H]
  \centering
  \includegraphics[width=0.5\textwidth]{example-image}
  \caption{A sample figure using the default image.}
  \label{fig:sample}
\end{figure}


\begin{table}[H]
\centering
\begin{tabular}{@{}lll@{}}
\toprule
Method & Accuracy & Time (s) \\
\midrule
A & 95\% & 1.23 \\
B & 93\% & 0.98 \\
\bottomrule
\end{tabular}
\caption{Sample comparison table.}
\label{tab:comparison}
\end{table}

\section{Citation Example}

As shown in Theorem~\ref{thm:sample}, the absolute value is always non-negative.

\section{Conclusion}
Lorem ipsum dolor sit amet. Suspendisse nec luctus dui. 

\bibliographystyle{plain}
\bibliography{refs}

\end{document}